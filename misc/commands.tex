%各个章节标题
\newcommand{\maintitle}{老天也会宠幸笨小孩}
\newcommand{\englishmaintitle}{God Loves Fools, too}
\newcommand{\chptitleOne}{绪论}
\newcommand{\chptitleTwo}{老天的万分之一也会宠幸到我们这些笨小孩}
\newcommand{\engchptitleTwo}{One millionth part of God's gift would also come to us stupid kids}

%自定义符号
\renewcommand{\algorithmicrequire}{\textbf{输入:}}
\renewcommand{\algorithmicensure}{\textbf{输出:}}
\DeclareMathOperator*{\argmax}{arg\,max}
\DeclareMathOperator*{\argmin}{arg\,min}
\DeclareMathOperator*{\softmax}{softmax}


%独立符号
\newcommand\indep{\protect\mathpalette{\protect\independenT}{\perp}}
\def\independenT#1#2{\mathrel{\rlap{$#1#2$}\mkern2mu{#1#2}}}

\renewcommand\arraystretch{1.5}%表格行高
\AtBeginEnvironment{algorithmic}{\setstretch{1.36}}%算法行距

\crefformat{algorithm}{算法~#2#1#3~}
\crefformat{table}{表~#2#1#3~}
\crefformat{figure}{图~#2#1#3~}
\crefformat{equation}{公式~(#2#1#3)~}
\crefformat{footnote}{$^{#2#1#3}$}
\crefformat{chapter}{章节~#2#1#3~}
\floatname{algorithm}{算法}

\AtBeginDocument{
  \crefformat{section}{章节~#2#1#3~}
  \crefformat{subsection}{章节~#2#1#3~}
  \crefformat{subsubsection}{章节~#2#1#3~}
}

\definecolor{bitred}{RGB}{161,62,11}
\definecolor{bitgreen}{RGB}{0,164,68}
\newcommand{\myredbox}[1]{\tikz[baseline=(MeNode.base)]{\node[rounded corners, fill=bitred!20](MeNode){#1};}}
\newcommand{\mygreenbox}[1]{\tikz[baseline=(MeNode.base)]{\node[rounded corners, fill=bitgreen!20](MeNode){#1};}}

